\documentclass[titlepage, 12pt, a4paper, oneside]{article}
\usepackage[utf8]{inputenc}
\usepackage[spanish, es-tabla]{babel}
\usepackage[T1]{fontenc}
\usepackage{hyperref}
\usepackage[numbib]{tocbibind}
\usepackage{tikz}
\usepackage[top=1in, bottom=1.25in, left=1.25in, right=1.25in]{geometry}
\usepackage{xcolor}

\usepackage{fancyhdr}
\pagestyle{fancy}
\fancyhf{}
\rhead{\textit{\color[rgb]{0.0,0.424,0.616}Nombre del estudiante}}
\lhead{}
\rfoot{}
\renewcommand{\headrulewidth}{0pt}

\usepackage{pgfgantt}

\title{}
\date{}
\renewcommand{\familydefault}{\sfdefault}

\begin{document}
\thispagestyle{empty}
\tikz[remember picture,overlay] \node[opacity=1.0,inner sep=0pt] at (current page.center){\includegraphics[width=\paperwidth,height=\paperheight]{Plantilla_AnteProyectoTFG-portada}};

\begin{center}
  \vspace{4cm}
  {\color{white} \Huge \textbf{Paralelismo de Tareas en el Intercomunicador intercomTM}}
\end{center}

\Large

\vspace{16.5ex}
\begin{tabular}{ll}
  ~~~~~~~~~ & Nombre del estudiante
\end{tabular}

\vspace{1.2cm}
\begin{tabular}{ll}
  ~~~~~~~~~ & Grado en Ingeniería Informática
\end{tabular}

\vspace{1.1cm}
\begin{tabular}{ll}
  ~~~~~~~~~ & Vicente González Ruiz
\end{tabular}

\vspace{1.2cm}
\begin{tabular}{ll}
  ~~~~~~~~~ & Departamento de Informática
\end{tabular}

\vspace{0.95cm}
\begin{tabular}{ll}
  ~~~~~~~~~ & Trabajo Técnico
\end{tabular}

\vspace{0.95cm}
\begin{tabular}{ll}
  ~~~~~~~~~ & Paralelismo de Tareas, Audio, Intercomunicador, \\
  ~~~~~~~~~ & Procesamiento en Tiempo Real
\end{tabular}

\clearpage

\tikz[remember picture,overlay] \node[opacity=1.0,inner sep=0pt] at (current page.center){\includegraphics[width=\paperwidth,height=\paperheight]{Plantilla_AnteProyectoTFG-paginas}};

\normalsize

\section{Introducción}
\label{sec:intro}
Un intercomunicador (intercom en inglés), es un dispositivo que
permite a dos o más usuarios comunicarse en tiempo real. Normalmente
sólo transmiten sonido, aunque también pueden transmitir imágenes
(vídeos). Se usan en los portales de las casas para saber quién está
llamando en la puerta, lo usan los motoristas para hablar entre ellos
cuando se desplazan y (lógicamente) llevan cascos, y se usan en
aplicaciones de comunicación a través de Internet, como Skype,
WhatsApp, etc.

El intercomunicador intercomTM es una aplicación desarrollada en la
Universidad de Almería por los alumnos de la asignatura de Tecnologías
Multimedia~\cite{intercomTM}, y posee las siguientes características:
\begin{enumerate}
\item Permite transmitir audio mono y estereo.
\item Permite elegir la frecuencia de muestreo.
\item Se adapta al ancho de banda disponible, enviando una señal de
  audio de calidad proporcional al ancho de banda.
\item Permite controlar la latencia.
\item Puede usarse en Internet.
\end{enumerate}

\section{Objetivos y justificación}
\label{sec:objetivos}
El objetivo de este TFG es mejorar la implementación actual de
intercomTM, aprovechando el paralelismo de
tareas~\cite{pacheco2011introduction}. Básicamente, se trata de
aprovechar las posibilidades multicore de la mayoría de las
computadoras actuales para incrementar el desempeño de intercomTM
(por ejemplo, permitiendo mayores tasa de muestreo en dispositivos con
poca potencia computacional, o aumentando las tasas de
compresión). Más concretamente, los objetivos que se desean alcanzar
son:
\begin{enumerate}
\item Reimplementar intercomTM usando el paradigma de programación de
  paralelismo de tareas, que básicamente consiste en descomponer el
  algoritmo usado (en intercomTM) en una serie de pasos
  (subalgoritmos) que se ejecuten concurrentemente, básicamente,
  formando un pipeline de tareas. Esto permitirá aprovechar un número
  de cores de nuestro dispositivo superior a 1 (si los hubiera).
\item Maximizar el balanceo de la carga de trabajo~\cite{workload}
  entre los diferentes cores, en función de los diferentes parámetros
  de funcionamiento que intercomTM posee.
\end{enumerate}

\section{Fases de desarrollo y estimación de su coste temporal}
\label{sec:fases}
Se preveen las siguientes unidades de trabajo y temporizaciones (véase
la Figura~\ref{fig:temporizacion}):
\begin{enumerate}
  \item {IDENTIFICACIÓN}: Identificación y comprensión de los
    requerimientos y objetivos planteados (10 h).
  \item {ANÁLISIS:} Estudio y análisis de la implementación actual de
    intercomTM, comprendiendo los algoritmos que utiliza (60 h).
  \item {APROVISIONAMIENTO}: Instalación y configuración del entorno
    de desarrollo y ejecución. Básicamente, Linux y Python (10h).
  \item {IMPLEMENTACIÓN}: Implementación de la versión paralela de
    intercomTM, en sucesivas versiones con niveles incrementales de
    paralelismo (60h).
  \item {EVALUACIÓN}: Evaluación de la solución y cuantificación del
    grado de consecución de los objetivos (30h).
  \item {REDACCIÓN}: Redacción de la memoria (30h).
\end{enumerate}

\begin{figure}
  \begin{center}
    \resizebox{\columnwidth}{!}{
      \begin{ganttchart}{1}{25}{10}
        \gantttitle{Bloques de 10 horas (total, 250 h)}{25} \\
        \gantttitlelist{1,...,25}{1} \\
        \ganttbar{IDENTIFICACIÓN}{1}{2} \\ % 50h
        \ganttlinkedbar{ANÁLISIS}{3}{9} \\ % 50 h
        \ganttlinkedbar{APROVISIONAMIENTO}{10}{11} \\ % 50 h
        \ganttlinkedbar{IMPLEMENTACIÓN}{12}{18} \\ % 50 h
        \ganttlinkedbar{EVALUACIÓN}{19}{22} \\ % 50 h
        \ganttlinkedbar{REDACCIÓN}{23}{25} % 50 h
      \end{ganttchart}
    }
  \end{center}
  \caption{Temporización del TFG.\label{fig:temporizacion}}
\end{figure}

\section{Requerimientos}
\label{sec:requerimientos}
Los requerimientos de la versión paralela de intercomTM son:

\begin{itemize}
\item Generales:
  \begin{enumerate}
  \item El lenguaje de programación será Python, y en concreto, se
    hará uso de la bilioteca multiprocessing~\cite{multiprocessing}.
  \item El estilo de programación debe seguir la norma
    PEP-8~\cite{PEP8}.
  \item El idioma usado, tanto para la definición de objetos, métodos,
    variables, etc., como para la documentación (incluyendo los
    comentarios en el código fuente) debe ser inglés.
  \item Se usará el paragigma de programación orientado a
    objetos~\cite{schach2008object} para ir obteniendo diferentes
    aproximaciones sucesivas del resultado final.
  \item El producto debe ejecutarse con éxito, al menos, en Linux,
    Windows y OSX.
  \end{enumerate}
\item Específicos:
  \begin{enumerate}
  \item Deberán aprovecharse tantos cores como sea posible, siempre y
    cuando esto no genere un decremento del desempeño.
  \item Mediante experimentación, deberá comprobarse que la versión
    paralela genera una carga máxima en los cores inferior a la carga
    que genera la versión secuencial (que sólo corre en un core), para
    un determinado conjunto (que puede cambiar) de parámetros de
    funcionamiento.
  \end{enumerate}
\end{itemize}

\section{Resultados esperados}
\label{sec:resultados}
Principalmente, se espera obtener la versión paralela de intercomTM.
Este deberá satisfacer los requerimientos (en definitiva, los
objetivos) planteados anteriormente, que se vefificarán mediante
experimentación.  Se definirán diferentes configuraciones típicas (por
ejemplo, (1) transmisión de audio mono, baja tasa de muestreo de 8000
Hz, y bajo ancho de banda (2) tranmisión estereo, alta tasa de
muestreo 48000 Hz, y ancho de banda no limitante) y se comprobará que
los requerimientos específicos se consiguen en la solución encontrada.

\section{Conclusiones esperadas}
\label{sec:conclusiones}
Básicamente, se tratará de concluir si el modelo de paralelismo de
tareas (ejecutadas mediante procesos) proporcionado por Python a
través de su API multiprocessing~\cite{multiprocessing} es adacuado (o
no) para conseguir los objetivos definidos (véase la
Sección~\ref{sec:objetivos}).

\bibliographystyle{plain}
\bibliography{../bibliografia}

\begin{center}
  \textbf{Firma del director (codirector)}
\end{center}

\end{document}
